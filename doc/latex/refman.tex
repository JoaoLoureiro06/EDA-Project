\documentclass[12pt]{article}
\usepackage[utf8]{inputenc}
\usepackage{amsmath, amssymb, amsfonts}
\usepackage{graphicx}
\usepackage{fancyhdr}
\usepackage{hyperref}
\usepackage{listings}
\usepackage{xcolor}
\usepackage{geometry}
\usepackage{titlesec}

% Configuração da página
\geometry{top=2.5cm, bottom=2.5cm, left=2.5cm, right=2.5cm}

% Configuração do código fonte
\lstset{
    language=C,
    basicstyle=\ttfamily\footnotesize,
    numbers=left,
    numberstyle=\tiny,
    stepnumber=1,
    numbersep=5pt,
    showstringspaces=false,
    captionpos=b,
    breaklines=true,
    frame=single,
    keywordstyle=\color{blue},
    stringstyle=\color{red},
    commentstyle=\color{green}\itshape,
}

% Configuração do cabeçalho e rodapé
\pagestyle{fancy}
\fancyhf{}
\fancyhead[L]{Gerenciamento de Antenas}
\fancyhead[C]{Trabalho de Estruturas de Dados}
\fancyhead[R]{2025}
\fancyfoot[C]{\thepage}

\title{\textbf{Gerenciamento de Antenas: Estruturas de Dados e Algoritmos}}
\author{Jo\~ao Loureiro}
\date{Mar\c{c}o de 2025}

\begin{document}

\maketitle

\begin{abstract}
Este trabalho apresenta um sistema para gerenciamento de antenas em uma cidade, utilizando listas ligadas simples. O sistema permite carregar, inserir, remover e listar antenas, além de detectar interferências e exibir um mapa representativo. Foram aplicadas técnicas de manipulação de listas dinâmicas para otimizar a eficiência das operações.
Link do repositorio Git: \href{https://github.com/JoaoLoureiro06/EDA-Project.git}{https://github.com/JoaoLoureiro06/EDA-Project.git}
\end{abstract}

\tableofcontents
\newpage

\section{Introdução}
O gerenciamento de antenas é um desafio na organização de redes de comunicação. Este projeto propõe um sistema baseado em estruturas de dados dinâmicas para armazenar e manipular informações sobre antenas, otimizando a detecção de interferências e o controle de posicionamento.

\section{Metodologia}
Utilizamos uma lista ligada simples para armazenar os dados das antenas. As operações implementadas incluem:
\begin{itemize}
    \item Inserção de novas antenas.
    \item Remoção de antenas existentes.
    \item Listagem de todas as antenas.
    \item Detecção de interferências com base na proximidade.
    \item Exibição de um mapa representando as antenas e áreas problemáticas.
\end{itemize}

\section{Estrutura de Dados Utilizada}
A estrutura básica da antena foi definida conforme abaixo:

\begin{lstlisting}[caption=Estrutura da Antena]
typedef struct Antena {
    char freq;
    int x, y;
    struct Antena* next;
} Antena;
\end{lstlisting}

Cada antena possui uma frequência representada por um caractere, além das coordenadas (x, y). A lista ligada armazena os elementos de maneira dinâmica.

\section{Implementação do Sistema}

\subsection{Criação e Inserção de Antenas}

\begin{lstlisting}[caption=Função para criar uma antena]
Antena* criarAntena(char freq, int x, int y) {
    Antena* nova = (Antena*)malloc(sizeof(Antena));
    nova->freq = freq;
    nova->x = x;
    nova->y = y;
    nova->next = NULL;
    return nova;
}
\end{lstlisting}

A inserção ocorre no início da lista:

\begin{lstlisting}[caption=Função para inserir antenas]
void inserirAntena(Antena** head, char freq, int x, int y) {
    Antena* nova = criarAntena(freq, x, y);
    nova->next = *head;
    *head = nova;
}
\end{lstlisting}

\subsection{Remoção de Antenas}

A remoção de antenas é realizada buscando pelo par de coordenadas:

\begin{lstlisting}[caption=Função para remover uma antena]
void removerAntena(Antena** head, int x, int y) {
    Antena* temp = *head, *prev = NULL;
    while (temp != NULL && (temp->x != x || temp->y != y)) {
        prev = temp;
        temp = temp->next;
    }
    if (temp == NULL) return;
    if (prev == NULL) *head = temp->next;
    else prev->next = temp->next;
    free(temp);
}
\end{lstlisting}

\subsection{Listagem de Antenas}
A listagem das antenas percorre a estrutura e exibe as informações:

\begin{lstlisting}[caption=Função para listar antenas]
void listarAntenas(Antena* head) {
    while (head != NULL) {
        printf("Frequência: %c | Posição: (%d, %d)\n", head->freq, head->x, head->y);
        head = head->next;
    }
}
\end{lstlisting}

\subsection{Detecção de Interferências}

A detecção ocorre comparando pares de antenas de mesma frequência:

\begin{lstlisting}[caption=Função para detectar interferências]
void deduzirEfeitoNefasto(Antena* head) {
    for (Antena* a1 = head; a1 != NULL; a1 = a1->next) {
        for (Antena* a2 = a1->next; a2 != NULL; a2 = a2->next) {
            if (a1->freq == a2->freq) {
                int mx = (a1->x + a2->x) / 2;
                int my = (a1->y + a2->y) / 2;
                printf("Efeito nefasto em (%d, %d)\n", mx, my);
            }
        }
    }
}
\end{lstlisting}

\section{Resultados e Discussão}
Os testes realizados mostraram que a lista ligada proporciona uma manipulação eficiente das antenas. A complexidade das operações principais é:
\begin{itemize}
    \item Inserção: $O(1)$
    \item Remoção: $O(n)$
    \item Listagem: $O(n)$
    \item Detecção de interferências: $O(n^2)$
\end{itemize}

Os resultados indicam que, para grandes volumes de dados, a detecção de interferências pode ser otimizada com estruturas adicionais, como árvores ou tabelas hash.

\section{Conclusão}
O sistema desenvolvido demonstrou eficiência na manipulação e análise de antenas. A estrutura de lista ligada possibilitou um controle dinâmico das antenas, mas melhorias podem ser implementadas para otimizar a detecção de interferências.

\section{Referências}
[1] Cormen, T. H., Leiserson, C. E., Rivest, R. L., Stein, C. \textit{Introduction to Algorithms}, MIT Press, 2009.

\end{document}
