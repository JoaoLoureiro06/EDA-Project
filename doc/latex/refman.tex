\documentclass[a4paper, 12pt]{article}
\usepackage[utf8]{inputenc}
\usepackage[T1]{fontenc}
\usepackage[portuguese]{babel}
\usepackage{geometry}
\usepackage{listings}
\usepackage{xcolor}
\usepackage{titlesec}
\usepackage{fancyhdr}
\usepackage{graphicx}
\usepackage{hyperref} 

\geometry{a4paper, left=25mm, right=25mm, top=30mm, bottom=25mm}

% Estilo para o código
\definecolor{codegreen}{rgb}{0,0.6,0}
\definecolor{codegray}{rgb}{0.5,0.5,0.5}
\definecolor{codepurple}{rgb}{0.58,0,0.82}
\definecolor{backcolour}{rgb}{0.95,0.95,0.92}

\lstdefinestyle{mystyle}{
    backgroundcolor=\color{backcolour},   
    commentstyle=\color{codegreen},
    keywordstyle=\color{magenta},
    numberstyle=\tiny\color{codegray},
    stringstyle=\color{codepurple},
    basicstyle=\ttfamily\footnotesize,
    breakatwhitespace=false,         
    breaklines=true,                 
    captionpos=b,                    
    keepspaces=true,                 
    numbers=left,                    
    numbersep=5pt,                  
    showspaces=false,                
    showstringspaces=false,
    showtabs=false,                  
    tabsize=2
}

\lstset{style=mystyle}

% Formatação de seções
\titleformat{\section}{\large\bfseries}{\thesection}{1em}{}
\titleformat{\subsection}{\normalsize\bfseries}{\thesubsection}{1em}{}

% Cabeçalho e rodapé
\pagestyle{fancy}
\fancyhf{}
\rhead{Sistema de Gerenciamento de Antenas}
\lhead{\thepage}
\renewcommand{\headrulewidth}{0.4pt}

% Capa
\title{
    \vspace{2cm}
    \textbf{Relatório Técnico} \\
    \large Estrutura de Dados Avançada \\
    \large Sistema de Gerenciamento de Antenas \\
    \vspace{1cm}
    \includegraphics[width=0.7\textwidth]{imagem.jpg} % Substitua por seu logo
    \vspace{3cm}
}
\author{
    Engenharia de Sistemas Informáticos\\
    Instituto Politécnico do Cavado e Ave  \\
    João Loureiro Nº31551 \\
    \vspace{1cm}
}
\date{\today}

\begin{document}

% Capa
\maketitle
\thispagestyle{empty}
\clearpage

% Sumário
\tableofcontents
\thispagestyle{empty}
\clearpage

\setcounter{page}{1}

\section{Introdução}
O Sistema de Gerenciamento de Antenas é uma aplicação desenvolvida em linguagem C para controle e monitoramento de antenas em um ambiente geográfico delimitado. O software permite:

\begin{itemize}
\item Inserção e remoção de antenas
\item Visualização em formato de mapa em formato matriz.
\item Detecção automática de pontos de interferência (efeito nefasto)
\item Carregamento e armazenamento de configurações
\end{itemize}

\section{Objetivos}
O sistema foi desenvolvido com os seguintes objetivos:

\begin{itemize}
\item Fornecer uma ferramenta eficiente para gestão de antenas
\item Identificar automaticamente áreas de interferência
\item Facilitar o planejamento de instalações
\item Oferecer visualização clara da disposição espacial
\end{itemize}

\section{Estrutura de Dados}
O sistema utiliza uma lista encadeada para armazenar as antenas:

\begin{lstlisting}[language=C]
typedef struct Antena {
    char freq;          // Frequencia (A-Z)
    int x, y;           // Coordenadas (0 a TAM_MAPA-1)
    struct Antena* next; // Ponteiro para proxima
} Antena;
\end{lstlisting}

\section{Funcionalidades Principais}

\subsection{Menu Principal}
Interface de controle do sistema:

\begin{lstlisting}[language=C]
int main() {
    Antena* lista = NULL;
    int opcao;
    char nomeArquivo[50];

    do {
        // Menu interativo
        printf("\n===== Menu =====\n");
        printf("1. Carregar Antenas de Arquivo\n");
        printf("2. Inserir Nova Antena\n");
        printf("3. Remover Antena\n");
        printf("4. Listar Antenas\n");
        printf("5. Deduzir Efeito Nefasto\n");
        printf("6. Mostrar Mapa\n");
        printf("7. Sair\n");
        
        // Lógica de seleção
        switch (opcao) {
            case 1: {
                // Carregar antenas a partir de um arquivo.
                printf("Digite o nome do arquivo: ");
                scanf("%s", nomeArquivo);
                carregarAntenasDoArquivo(&lista, nomeArquivo);
                break;
            }
            case 2: {
                // Inserir uma nova antena.
                char freq; ///< Frequência da antena.
                int x, y; ///< Posições da antena (X, Y).
                printf("Frequência da antena (letra): ");
                scanf(" %c", &freq); ///< Lê a frequência da antena.
                printf("Posição X: ");
                scanf("%d", &x); ///< Lê a posição X.
                printf("Posição Y: ");
                scanf("%d", &y); ///< Lê a posição Y.

                // Verifica se as posições estão dentro dos limites do mapa.
                if (x >= 0 && x < TAM_MAPA && y >= 0 && y < TAM_MAPA) {
                    inserirAntena(&lista, freq, x, y);
                    printf("Antena inserida!\n");
                } else {
                    printf("Posição inválida!\n");
                }
                break;
            }
            case 3: {
                // Remover uma antena da lista.
                int x, y; ///< Posições da antena a ser removida.
                printf("Posição X da antena a remover: ");
                scanf("%d", &x); ///< Lê a posição X da antena.
                printf("Posição Y da antena a remover: ");
                scanf("%d", &y); ///< Lê a posição Y da antena.
                removerAntena(&lista, x, y);
                break;
            }
            case 4:
                // Listar todas as antenas.
                listarAntenas(lista);
                break;
            case 5:
                // Deduzir as localizações com efeito nefasto.
                deduzirEfeitoNefasto(lista);
                break;
            case 6:
                // Mostrar o mapa com as antenas.
                mostrarMapa(lista);
                break;
            case 7:
                // Sair do programa.
                printf("Saindo...\n");
                break;
            default:
                // Caso o usuário insira uma opção inválida.
                printf("Opção inválida! Tente novamente.\n");
        }
    } while (opcao != 7);
    
    liberarLista(lista);
    return 0;
}
\end{lstlisting}

\subsection{Gerenciamento de Antenas}
 Inserção e Remoção de Antenas:

\begin{lstlisting}[language=C]
void inserirAntena(Antena** head, char freq, int x, int y) {
    if (x < 0 || x >= TAM_MAPA || y < 0 || y >= TAM_MAPA) {
        printf("Posicao invalida!\n");
        return;
    }
    Antena* nova = criarAntena(freq, x, y);
    nova->next = *head;
    *head = nova;
}

void removerAntena(Antena** head, int x, int y) {
    Antena* temp = *head;
    Antena* prev = NULL;

    while (temp != NULL && (temp->x != x || temp->y != y)) {
        prev = temp;
        temp = temp->next;
    }

    if (temp == NULL) {
        printf("Antena não encontrada na posição (%d, %d)!\n", x, y);
        return;
    }

    if (prev == NULL) { // Primeira da lista
        *head = temp->next;
    } else {
        prev->next = temp->next;
    }

    free(temp);
    printf("Antena removida com sucesso!\n");
}
\end{lstlisting}

\section{Detecção de Interferência}
Algoritmo para identificar pontos de efeito nefasto:

\begin{lstlisting}[language=C]
  void deduzirEfeitoNefasto(Antena* head) {
    char mapa[TAM_MAPA][TAM_MAPA];

    // Inicializa o mapa com '.'
    for (int i = 0; i < TAM_MAPA; i++) {
        for (int j = 0; j < TAM_MAPA; j++) {
            mapa[i][j] = '.';
        }
    }

    // Coloca as antenas no mapa
    Antena* temp = head;
    while (temp != NULL) {
        mapa[temp->x][temp->y] = temp->freq;
        temp = temp->next;
    }

    // Verifica pares de antenas para interferência
    for (Antena* ant1 = head; ant1 != NULL; ant1 = ant1->next) {
        for (Antena* ant2 = ant1->next; ant2 != NULL; ant2 = ant2->next) {
            if (ant1->freq == ant2->freq) { // Mesma frequência
                int dx = ant2->x - ant1->x;
                int dy = ant2->y - ant1->y;

                // Confere alinhamento (horizontal, vertical, diagonal)
                if ((dx == 0 || dy == 0 || abs(dx) == abs(dy))) {
                    int mx = (ant1->x + ant2->x) / 2;
                    int my = (ant1->y + ant2->y) / 2;

                    // Verifica se o ponto médio está dentro dos limites do mapa e não é uma antena
                    if (mx >= 0 && mx < TAM_MAPA && my >= 0 && my < TAM_MAPA && mapa[mx][my] == '.') {
                        mapa[mx][my] = '#'; // Marca o efeito nefasto
                    }
                }
            }
        }
    }

    // Mostra o mapa com efeito nefasto
    printf("\nMapa com Localizações com Efeito Nefasto:\n");
    for (int i = 0; i < TAM_MAPA; i++) {
        for (int j = 0; j < TAM_MAPA; j++) {
            printf("%c", mapa[i][j]);
        }
        printf("\n");
    }

}
\end{lstlisting}

\section{Conclusão}
O sistema desenvolvido apresenta:

\begin{itemize}
\item Arquitetura modular e bem estruturada
\item Algoritmos eficientes para detecção de interferências
\item Interface intuitiva e fácil de usar
\item Código otimizado e com boa documentação
\end{itemize}

Como trabalhos futuros, recomenda-se:

\begin{itemize}
\item Implementação de persistência em banco de dados
\item Adição de interface gráfica
\item Estudos de performance para grandes volumes de dados
\item Repositório GitHub: \href{https://github.com/JoaoLoureiro06/EDA-Project.git}{Clique aqui para acessar o repositório}
\end{itemize}

\end{document}